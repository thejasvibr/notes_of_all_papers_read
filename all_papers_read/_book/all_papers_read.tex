% Options for packages loaded elsewhere
\PassOptionsToPackage{unicode}{hyperref}
\PassOptionsToPackage{hyphens}{url}
%
\documentclass[
]{book}
\usepackage{lmodern}
\usepackage{amssymb,amsmath}
\usepackage{ifxetex,ifluatex}
\ifnum 0\ifxetex 1\fi\ifluatex 1\fi=0 % if pdftex
  \usepackage[T1]{fontenc}
  \usepackage[utf8]{inputenc}
  \usepackage{textcomp} % provide euro and other symbols
\else % if luatex or xetex
  \usepackage{unicode-math}
  \defaultfontfeatures{Scale=MatchLowercase}
  \defaultfontfeatures[\rmfamily]{Ligatures=TeX,Scale=1}
\fi
% Use upquote if available, for straight quotes in verbatim environments
\IfFileExists{upquote.sty}{\usepackage{upquote}}{}
\IfFileExists{microtype.sty}{% use microtype if available
  \usepackage[]{microtype}
  \UseMicrotypeSet[protrusion]{basicmath} % disable protrusion for tt fonts
}{}
\makeatletter
\@ifundefined{KOMAClassName}{% if non-KOMA class
  \IfFileExists{parskip.sty}{%
    \usepackage{parskip}
  }{% else
    \setlength{\parindent}{0pt}
    \setlength{\parskip}{6pt plus 2pt minus 1pt}}
}{% if KOMA class
  \KOMAoptions{parskip=half}}
\makeatother
\usepackage{xcolor}
\IfFileExists{xurl.sty}{\usepackage{xurl}}{} % add URL line breaks if available
\IfFileExists{bookmark.sty}{\usepackage{bookmark}}{\usepackage{hyperref}}
\hypersetup{
  pdftitle={Notes of all papers read so far},
  pdfauthor={Thejasvi Beleyur},
  hidelinks,
  pdfcreator={LaTeX via pandoc}}
\urlstyle{same} % disable monospaced font for URLs
\usepackage{longtable,booktabs}
% Correct order of tables after \paragraph or \subparagraph
\usepackage{etoolbox}
\makeatletter
\patchcmd\longtable{\par}{\if@noskipsec\mbox{}\fi\par}{}{}
\makeatother
% Allow footnotes in longtable head/foot
\IfFileExists{footnotehyper.sty}{\usepackage{footnotehyper}}{\usepackage{footnote}}
\makesavenoteenv{longtable}
\usepackage{graphicx,grffile}
\makeatletter
\def\maxwidth{\ifdim\Gin@nat@width>\linewidth\linewidth\else\Gin@nat@width\fi}
\def\maxheight{\ifdim\Gin@nat@height>\textheight\textheight\else\Gin@nat@height\fi}
\makeatother
% Scale images if necessary, so that they will not overflow the page
% margins by default, and it is still possible to overwrite the defaults
% using explicit options in \includegraphics[width, height, ...]{}
\setkeys{Gin}{width=\maxwidth,height=\maxheight,keepaspectratio}
% Set default figure placement to htbp
\makeatletter
\def\fps@figure{htbp}
\makeatother
\setlength{\emergencystretch}{3em} % prevent overfull lines
\providecommand{\tightlist}{%
  \setlength{\itemsep}{0pt}\setlength{\parskip}{0pt}}
\setcounter{secnumdepth}{5}
\usepackage{booktabs}
\usepackage[]{natbib}
\bibliographystyle{apalike}

\title{Notes of all papers read so far}
\author{Thejasvi Beleyur}
\date{Last Updated : 2020-07-21}

\begin{document}
\maketitle

{
\setcounter{tocdepth}{1}
\tableofcontents
}
\hypertarget{what-this-book-is-about}{%
\chapter*{What this book is about}\label{what-this-book-is-about}}
\addcontentsline{toc}{chapter}{What this book is about}

This book is a compilation of all the notes I will be making for the papers that I read from now.
I've been realising my paper notes are scattered everywhere, across multiple folders and multiple computers,
this is my attempt at trying to unify everything into one place.

\hypertarget{toledo-et-al.-2020-science}{%
\chapter{Toledo et al.~2020, Science}\label{toledo-et-al.-2020-science}}

\chaptermark{cognitive maps in bats with high throughput tracking}

\emph{Cognitive map--based navigation in wild bats revealed by a new high-throughput tracking system.} \citep{toledo2020cognitive}

\begin{itemize}
\tightlist
\item
  \emph{notes taken on 2020-07-14}
\end{itemize}

\hypertarget{introduction}{%
\section{Introduction}\label{introduction}}

\begin{itemize}
\tightlist
\item
  map based navigation goes beyond just simple modes eg. beacon following or landmark based navigation.
\item
  bats are known to return to their normal sites even after displacement, which suggests `map-and-compass' navigation style
\item
  Authors' previous results showed that fruit bats flew straight paths, but this was limited to a few nights of data.
\item
  In this study, authors managed to study 172 bats over a cumulative of 3449 nights.
\end{itemize}

\hypertarget{methods}{%
\section{Methods}\label{methods}}

\begin{itemize}
\tightlist
\item
  \emph{ATLAS} - a reverse GPS system, where the animal wears a tag that emits a signal which is received by multiple ground stations - and thus using TOADs, the animal can be detected.
\item
  ATLAS coverage region is \textasciitilde88,200 hectares (or an area that's 29X29km big!!)
\item
  bats tagged, and all fruit trees within a given region recorded.
\item
  Also performed translocation experiments. Each bat was translocated to the periphery of its normal foraging area, but within detection range of their foraging area
\item
  Also performed time-lag embedding to understand how complex the navigational mechanism is
\end{itemize}

\hypertarget{results}{%
\section{Results}\label{results}}

\begin{itemize}
\item
  Bats exhibited straight tracks, which is indicative of goal-directed behabiou
\item
  Each bat had its favourite tree, and visited it every night, and even visited it from multiple directions of arrival
\item
  Solid evidence for a cognitive map is when an animal moves between two points that can't be detected/seen/observed from each other (ie. it requires a kind of `rigorous' mapping)
\item
  4.3\% of all tracks,and 70/172 bats actually showed such shortcuts
\item
  There was no difference in the rate at which shortcuts happened between the age groups of bats tagged
\item
  \emph{following a conspecific} -- they talk about it by saying that in their dataset, they didn't see individuals flying close together - but they only tagged 172 bats of ??? thousand in the whole population
\item
  Translocated bats were able to return to their normal foraging area
\item
  Time-lag embedding showed a high-dimensional correlation (?) indicating there must be many difference navigational factors

  \begin{itemize}
  \item
    If bats were following a simple navigational route, they might always arrive and depart from the same direction - but the authors don't see this.
  \item
    Authors don't seem convinced about the idea of an olfactory map
  \item
    Authors also rule out the idea of pure path integration because they show that many bats returned to a different cave than the one they started out the evening from
  \item ~
    \hypertarget{comments}{%
    \subsection{Comments}\label{comments}}
  \end{itemize}
\item
  Fig 2E: why would you use the p-value to show the \emph{absence} of an effect? The p-value per se is hinged on so many other factors (eg.power, effect size, sample size), why not just report the raw data
\item
\end{itemize}

\hypertarget{harten-et-al.-2020-science}{%
\chapter{Harten et al.~2020, Science}\label{harten-et-al.-2020-science}}

\chaptermark{The ontogeny of a bat cognitive map}

\emph{The ontogeny of a mammalian cognitive map in the real world} \citep{harten2020ontogeny}

\begin{itemize}
\tightlist
\item
  \emph{notes taken on 2020-07-16}
\end{itemize}

\hypertarget{introduction-methods-results}{%
\section{Introduction, Methods, Results}\label{introduction-methods-results}}

\begin{itemize}
\item
  Whether animals navigate using 'maps'or not remains a question. The ability to take shortcuts, or direct routes between two points is a hallmark of map based navigation.
\item
  The main problem with studying animal namvigation in the wild is that we can never be sure that the animal has not taken an apparent `shortcut' before.
\item
  authors were able to GPS track 22 young Egyptian Fruit Bats (\emph{Rousettus aegeyptiacus}) from their first flight out of the roost
\item
  Young bats increased their home range over the course of \textasciitilde70 nights, by which they had the same home range size as an adult.
\item
  Individuals showed two types of broad flight behaviour over a night, `exploratory', where they explored for new trees, and nights where they visited prviously visted trees.
\item
  Evidence to support the fact that the shortcuts were intentional:

  \begin{itemize}
  \tightlist
  \item
    shortcut were as straight as familiar routes (`commutes')
  \item
    individuals seem to head in the direction of their target from the start of the `shortcut'
  \item
    the `shortcuts' could not be replicated by a random correlated walk (\emph{this seems like a bit of a straw man null model (\href{https://www.fharrell.com/post/nhst-never/}{link}), especially since the data is clearly so directional. The authors also specifically mention `but without any navigational goal'})
  \end{itemize}
\item
  \emph{Bats performed both shortcuts and long-cuts from their first day outside,} - this is pretty impressive, but this also makes me think that the bats may actually be relying on a kind of path integration. Is their apatial memory so good that they can start mapping things over the course of one night? Is it possible that the bats may actually be using a beacon-type stategy to find their way around?
\item
  Authors rule out olfaction and sound based cues by comparing wind direction and actual recordings made on the backpack tags. Both don't show support.
\item
  \emph{`Bats that were closer to the translocation release point before the translocation night did not necessarily navigate home better, once again contradicting the template-matching hypothesis'} , the authors also go on to follow and say that bats that flew higher were better able to find their way back. This actually doesn't rule out the template matching hypothesis either, because it might just mean that bats that flew higher had larger access to the area below, to form a `higher SNR' template perhaps\ldots{}
\item
  The authors do also admit that the navigation behaviou they observed may be a result of multiple navigation strategies: \emph{`\ldots, navigation is a complex behavior that probably does not always rely on a single strategy'}
\item
\end{itemize}

\hypertarget{comments}{%
\subsection{Comments}\label{comments}}

\begin{itemize}
\item
  `How animals navigate over large-scale environments remains a riddle', first line of the abstract starts with a rather bold statement. Is this statement really true for all animals, am under the impression that there is a large body of work for at least some animals.
\item
  `\emph{We documented how young pupsdeveloped their visual-based map}' - interesting, does this mean, echolocation develops later, or that the bats are known to use primarily vision for their navigation?
\item
  Remember listening to a talk by Lee Harten in Konstanz ASAB, where she also presented work on the flight behaviour of the mother, who carried her pups around, and how the mother used to leave the pup in one tree, and so on. Do the authors discuss the implications of this type of memory on the shortcut taking ability of the young bats? \emph{Yes, the authors have taken care of this, in the SI, they clearly state the mother and the pups were brought into an indoor facility, and the pups were kept indoors until they could fly}
\item
  \emph{What about bats flying together?}, or encountering each other? Their inhourse colony data kind of excludes this idea because the individual bats arrive alone, and are spaced b a few minutes. This is not the most rigorous evidence, but is still pretty indicatibe, eg. even in Orlova Chuka (and other caves), you can see the bats arrive alone in the morning. Howeer, this still doesn't really exclude the fact that bats may be encountering each other at some point over the course of the night.
\item
\end{itemize}

\hypertarget{wikelskiearthquake}{%
\chapter{Wikelski et al.~2020}\label{wikelskiearthquake}}

\emph{Potential short-term eathquake forecasting by farm animal monitoring} \citep{wikelskiearthquake2020}

\begin{itemize}
\item
  weird animal behaviour just before earthquakes habe been reported, including dramatic cases where snakes and rats came out of their winter burrows during the winter in the 1975 Haicheng earthquake
\item
  finding reliable changes in animal behaviour is tough because animal the animal behaviour needs to be monitored before and during the earthquake.
\item
  Authors were able to overcome some of the limitations in the data this time by tagging multiple farm animals with high-resolution GPS tags that were equipped with many types of sensors
\item
  Authors measured the behaviour of animals at the M6.6 Norcia earthquake that happened on 2016
\end{itemize}

\hypertarget{methods}{%
\section{Methods}\label{methods}}

\begin{itemize}
\item
  Animals chosen from a farm based on which ones the animals thought were most sensitive to the earthquakes.
\item
  Two tagging periods, once before + during the earthquake, once after
\end{itemize}

\hypertarget{data-description}{%
\section{Data description}\label{data-description}}

\begin{itemize}
\item
  \emph{Between \ldots, the animals experienced a total of 5,304 earthquakes with M \textgreater{} 0.4 \ldots and from \ldots{} a total of 12,948} \ldots{} didn't realise that earthquakes were so frequent in some areas.
\item
  The `hypocenters' of the earthquakes were anywhere between 5-28 km from the farm --\emph{all relatively close by!}
\end{itemize}

\hypertarget{results}{%
\subsection{Results}\label{results}}

\begin{itemize}
\item
  Find a negative correlation between time of increased animal activity and earthquake intensity. For earthquakes \(\geq\) 4 M, the animals responded earlier to quakes that were closer to the farm, and later to those that were further away from the farm.
\item
  ``Warning times'' ranged from 1-15 hours
\item
  Animals seemed to be more sensitive to earthquakes in closed buildings - but there may be a seasonal factor in the observations too
\item
  What are the possible cues the animals are using to detect/respond to these earthquakes

  \begin{itemize}
  \tightlist
  \item
    The inverse relation hints at a diffusive type process. ``air ionization at pressurised rock surfaces'' -- diffusing into the air, to which the animals may be responding to.
  \end{itemize}
\end{itemize}

\hypertarget{overall-thoughts}{%
\subsection{Overall thoughts}\label{overall-thoughts}}

\begin{itemize}
\item
  very intersting paper, which quantifies something which has been known but has now been studied in greater detail through this new technology.
\item
  authors also suggest a future experimental setup where a series of animal moinorting stations could be used to predict the position and time of arrival of an earthquake.
\end{itemize}

\hypertarget{ratcliffe-et-al.-2004-can.-j.-zool.}{%
\chapter{Ratcliffe et al.~2004, Can. J. Zool.}\label{ratcliffe-et-al.-2004-can.-j.-zool.}}

\chaptermark{Conspecifics influence call design}

\emph{Conspecifics influence call design in the Brazlilan free-tailed bat, }Tadarida brasiliensis* * \citep{ratcliffe2004conspecifics}

\begin{itemize}
\tightlist
\item
  \emph{notes taken on 2020-07-21}
\end{itemize}

\hypertarget{introduction}{%
\section{Introduction}\label{introduction}}

\begin{itemize}
\tightlist
\item
  echolocation is pretty flexible and the emitted calls vary a lot based on the type of prey being caught, and the presence of conspecifics
\item
  This paper is a kind of offspring of another (Avila-Flores 2003), where the authors saw that there was more call variation when bats flew together than when alone.
\item
  Authors try to estimate this variation by comparing observed pairs of bat calls and `virtual' pairs of bat calls.
\end{itemize}

\hypertarget{methods}{%
\section{Methods}\label{methods}}

\begin{itemize}
\tightlist
\item
  Free flying \emph{Tadarida brasiliensis} recorded in three locations in Mexico City, of bats flying over a lake, and two others in open areas (park and city square)
\item
  30 sequences from each location obtained (15 single bat passes + 15 paired passes).
\item
  Authors specifically chose call sequences with no overlap of bat calls.
\end{itemize}

\hypertarget{analysis}{%
\section{Analysis}\label{analysis}}

\begin{itemize}
\tightlist
\item
  To compare the single vs paired call behaviour, the difference in the mean values was used
\item
  The difference in the mean, \(\Delta_{mean}\) was used to compare if a bat showed alterations to its echolocation when alone vs.~when with another bats.
\item
  To understand if bats actually altered their call parameters when flying in groups the authors calculated the pair difference \(bat1_{measurement}-bat2_{measurement}\) for actual observed pairs of bat call sequences, and those of virtual pairs.
\item
  The pair difference was calculated by subtracting the means of the \(bat1_{measurement}\) and \(bat2_{measurement}\)
\end{itemize}

\hypertarget{results}{%
\section{Results}\label{results}}

\begin{itemize}
\tightlist
\item
  Authors didn't find any statistical difference in call parameters across locations and so decided to pool them all together (and thus used single bat call sequences from multiple locations while making virtual pairs). (\emph{See comments})
\item
  Authors found no inter-individual call differences across observed and virtual pairs for the parameters 1) call duration 2) bandwidth , and found differences in the peak frequency. (\emph{See comments})
\item
  The authors also saw social calls in paired audio files.
\item
  The authors are indeed careful while ending `\emph{To be even-handed\ldots.changes in peak frequency which we found\ldots are neither jamming avoidance nor air traffic control, but serve another and as yet undetermined communicative function}'
\end{itemize}

\hypertarget{comments}{%
\subsection{Comments}\label{comments}}

\begin{itemize}
\item
  authors state \emph{`Although referred to as jamming avoidance.., support for this interpretation is not as strong as that for the jamming avoidance response in electric fish'}. They also go on to state that labortatory studies seem to have `met with some success'. In general, does this support the fact that bats in the field may actually not be showing dramatic changes in calls at all - it's only when they're put into a quiet unusual lab context that they begin to show changes. Perhaps, this strong response is seen because the animals have gotten used to flying under quiest conditions, and are now suddenly challenged?
\item
  how do the authors actually know that the two bats in a paired call sequence were flying together/in close proximity? `\emph{We assinged each recorded seequence to one of two situations\ldots..and two bats flying close proximity}' - this is a rather vague definition. With a single bat detector, it is not possible to track the bats in 3d, at most one can check if the waveforms of the two bat calls are similar, but here too it only means that the two bats were flying in the same radius from the bat detector. It is somewhat crude, though admittedly, the best possible criteria given the current instrument.
\item
  Authors actually pooled call sequences from multiple locations and made virtual call pairs from this pool. This will actually have the effect of \emph{increasing} the diversity of the observed data? Also, it is known from tracking studies that individual bats tend of have favourit foraging sites - and so in some sense, it is expected that there will be a `local' flavour to the data. It would have been nice to see the authors perform the same analysis without pooling across locations.
\item
  The authors find differences in peak frequency between real and virtual pairs. This could really be an effect of where the bats were flying while in pairs vs when they were alone. The simplest explanation is that the bats may be flying further away from the microphone, thus leading to different SNRs at recording - which then leads to different peak frequencies. The difference in the peak frequency could really be an artifact and not a real effect. Moreover, given the authors state that they do see social calls, the line between social call and echolocation call is a thin one, which means, perhaps the authors are seeing this effect? The authors state this themselves \emph{`Our recordings suggest a continuum in call features between echolocation and social calls\ldots{}'}
\item
\end{itemize}

\hypertarget{final-words}{%
\chapter{Final Words}\label{final-words}}

We have finished a nice book.

  \bibliography{book.bib}

\end{document}
