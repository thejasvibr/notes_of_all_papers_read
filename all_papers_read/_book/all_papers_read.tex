% Options for packages loaded elsewhere
\PassOptionsToPackage{unicode}{hyperref}
\PassOptionsToPackage{hyphens}{url}
%
\documentclass[
]{book}
\usepackage{lmodern}
\usepackage{amssymb,amsmath}
\usepackage{ifxetex,ifluatex}
\ifnum 0\ifxetex 1\fi\ifluatex 1\fi=0 % if pdftex
  \usepackage[T1]{fontenc}
  \usepackage[utf8]{inputenc}
  \usepackage{textcomp} % provide euro and other symbols
\else % if luatex or xetex
  \usepackage{unicode-math}
  \defaultfontfeatures{Scale=MatchLowercase}
  \defaultfontfeatures[\rmfamily]{Ligatures=TeX,Scale=1}
\fi
% Use upquote if available, for straight quotes in verbatim environments
\IfFileExists{upquote.sty}{\usepackage{upquote}}{}
\IfFileExists{microtype.sty}{% use microtype if available
  \usepackage[]{microtype}
  \UseMicrotypeSet[protrusion]{basicmath} % disable protrusion for tt fonts
}{}
\makeatletter
\@ifundefined{KOMAClassName}{% if non-KOMA class
  \IfFileExists{parskip.sty}{%
    \usepackage{parskip}
  }{% else
    \setlength{\parindent}{0pt}
    \setlength{\parskip}{6pt plus 2pt minus 1pt}}
}{% if KOMA class
  \KOMAoptions{parskip=half}}
\makeatother
\usepackage{xcolor}
\IfFileExists{xurl.sty}{\usepackage{xurl}}{} % add URL line breaks if available
\IfFileExists{bookmark.sty}{\usepackage{bookmark}}{\usepackage{hyperref}}
\hypersetup{
  pdftitle={Notes of all papers read so far},
  pdfauthor={Thejasvi Beleyur},
  hidelinks,
  pdfcreator={LaTeX via pandoc}}
\urlstyle{same} % disable monospaced font for URLs
\usepackage{longtable,booktabs}
% Correct order of tables after \paragraph or \subparagraph
\usepackage{etoolbox}
\makeatletter
\patchcmd\longtable{\par}{\if@noskipsec\mbox{}\fi\par}{}{}
\makeatother
% Allow footnotes in longtable head/foot
\IfFileExists{footnotehyper.sty}{\usepackage{footnotehyper}}{\usepackage{footnote}}
\makesavenoteenv{longtable}
\usepackage{graphicx,grffile}
\makeatletter
\def\maxwidth{\ifdim\Gin@nat@width>\linewidth\linewidth\else\Gin@nat@width\fi}
\def\maxheight{\ifdim\Gin@nat@height>\textheight\textheight\else\Gin@nat@height\fi}
\makeatother
% Scale images if necessary, so that they will not overflow the page
% margins by default, and it is still possible to overwrite the defaults
% using explicit options in \includegraphics[width, height, ...]{}
\setkeys{Gin}{width=\maxwidth,height=\maxheight,keepaspectratio}
% Set default figure placement to htbp
\makeatletter
\def\fps@figure{htbp}
\makeatother
\setlength{\emergencystretch}{3em} % prevent overfull lines
\providecommand{\tightlist}{%
  \setlength{\itemsep}{0pt}\setlength{\parskip}{0pt}}
\setcounter{secnumdepth}{5}
\usepackage{booktabs}
\usepackage[]{natbib}
\bibliographystyle{apalike}

\title{Notes of all papers read so far}
\author{Thejasvi Beleyur}
\date{Last Updated : 2020-08-04}

\begin{document}
\maketitle

{
\setcounter{tocdepth}{1}
\tableofcontents
}
\hypertarget{what-this-book-is-about}{%
\chapter*{What this book is about}\label{what-this-book-is-about}}
\addcontentsline{toc}{chapter}{What this book is about}

This book is a compilation of all the notes I will be making for the papers that I read from now.
I've been realising my paper notes are scattered everywhere, across multiple folders and multiple computers,
this is my attempt at trying to unify everything into one place.

\hypertarget{toledo-et-al.-2020-science}{%
\chapter{Toledo et al.~2020, Science}\label{toledo-et-al.-2020-science}}

\chaptermark{cognitive maps in bats with high throughput tracking}

\emph{Cognitive map--based navigation in wild bats revealed by a new high-throughput tracking system.} \citep{toledo2020cognitive}

\begin{itemize}
\tightlist
\item
  \emph{notes taken on 2020-07-14}
\end{itemize}

\hypertarget{introduction}{%
\section{Introduction}\label{introduction}}

\begin{itemize}
\tightlist
\item
  map based navigation goes beyond just simple modes eg. beacon following or landmark based navigation.
\item
  bats are known to return to their normal sites even after displacement, which suggests `map-and-compass' navigation style
\item
  Authors' previous results showed that fruit bats flew straight paths, but this was limited to a few nights of data.
\item
  In this study, authors managed to study 172 bats over a cumulative of 3449 nights.
\end{itemize}

\hypertarget{methods}{%
\section{Methods}\label{methods}}

\begin{itemize}
\tightlist
\item
  \emph{ATLAS} - a reverse GPS system, where the animal wears a tag that emits a signal which is received by multiple ground stations - and thus using TOADs, the animal can be detected.
\item
  ATLAS coverage region is \textasciitilde88,200 hectares (or an area that's 29X29km big!!)
\item
  bats tagged, and all fruit trees within a given region recorded.
\item
  Also performed translocation experiments. Each bat was translocated to the periphery of its normal foraging area, but within detection range of their foraging area
\item
  Also performed time-lag embedding to understand how complex the navigational mechanism is
\end{itemize}

\hypertarget{results}{%
\section{Results}\label{results}}

\begin{itemize}
\item
  Bats exhibited straight tracks, which is indicative of goal-directed behabiou
\item
  Each bat had its favourite tree, and visited it every night, and even visited it from multiple directions of arrival
\item
  Solid evidence for a cognitive map is when an animal moves between two points that can't be detected/seen/observed from each other (ie. it requires a kind of `rigorous' mapping)
\item
  4.3\% of all tracks,and 70/172 bats actually showed such shortcuts
\item
  There was no difference in the rate at which shortcuts happened between the age groups of bats tagged
\item
  \emph{following a conspecific} -- they talk about it by saying that in their dataset, they didn't see individuals flying close together - but they only tagged 172 bats of ??? thousand in the whole population
\item
  Translocated bats were able to return to their normal foraging area
\item
  Time-lag embedding showed a high-dimensional correlation (?) indicating there must be many difference navigational factors

  \begin{itemize}
  \item
    If bats were following a simple navigational route, they might always arrive and depart from the same direction - but the authors don't see this.
  \item
    Authors don't seem convinced about the idea of an olfactory map
  \item
    Authors also rule out the idea of pure path integration because they show that many bats returned to a different cave than the one they started out the evening from
  \item ~
    \hypertarget{comments}{%
    \subsection{Comments}\label{comments}}
  \end{itemize}
\item
  Fig 2E: why would you use the p-value to show the \emph{absence} of an effect? The p-value per se is hinged on so many other factors (eg.power, effect size, sample size), why not just report the raw data
\item
\end{itemize}

\hypertarget{harten-et-al.-2020-science}{%
\chapter{Harten et al.~2020, Science}\label{harten-et-al.-2020-science}}

\chaptermark{The ontogeny of a bat cognitive map}

\emph{The ontogeny of a mammalian cognitive map in the real world} \citep{harten2020ontogeny}

\begin{itemize}
\tightlist
\item
  \emph{notes taken on 2020-07-16}
\end{itemize}

\hypertarget{introduction-methods-results}{%
\section{Introduction, Methods, Results}\label{introduction-methods-results}}

\begin{itemize}
\item
  Whether animals navigate using 'maps'or not remains a question. The ability to take shortcuts, or direct routes between two points is a hallmark of map based navigation.
\item
  The main problem with studying animal namvigation in the wild is that we can never be sure that the animal has not taken an apparent `shortcut' before.
\item
  authors were able to GPS track 22 young Egyptian Fruit Bats (\emph{Rousettus aegeyptiacus}) from their first flight out of the roost
\item
  Young bats increased their home range over the course of \textasciitilde70 nights, by which they had the same home range size as an adult.
\item
  Individuals showed two types of broad flight behaviour over a night, `exploratory', where they explored for new trees, and nights where they visited prviously visted trees.
\item
  Evidence to support the fact that the shortcuts were intentional:

  \begin{itemize}
  \tightlist
  \item
    shortcut were as straight as familiar routes (`commutes')
  \item
    individuals seem to head in the direction of their target from the start of the `shortcut'
  \item
    the `shortcuts' could not be replicated by a random correlated walk (\emph{this seems like a bit of a straw man null model (\href{https://www.fharrell.com/post/nhst-never/}{link}), especially since the data is clearly so directional. The authors also specifically mention `but without any navigational goal'})
  \end{itemize}
\item
  \emph{Bats performed both shortcuts and long-cuts from their first day outside,} - this is pretty impressive, but this also makes me think that the bats may actually be relying on a kind of path integration. Is their apatial memory so good that they can start mapping things over the course of one night? Is it possible that the bats may actually be using a beacon-type stategy to find their way around?
\item
  Authors rule out olfaction and sound based cues by comparing wind direction and actual recordings made on the backpack tags. Both don't show support.
\item
  \emph{`Bats that were closer to the translocation release point before the translocation night did not necessarily navigate home better, once again contradicting the template-matching hypothesis'} , the authors also go on to follow and say that bats that flew higher were better able to find their way back. This actually doesn't rule out the template matching hypothesis either, because it might just mean that bats that flew higher had larger access to the area below, to form a `higher SNR' template perhaps\ldots{}
\item
  The authors do also admit that the navigation behaviou they observed may be a result of multiple navigation strategies: \emph{`\ldots, navigation is a complex behavior that probably does not always rely on a single strategy'}
\item
\end{itemize}

\hypertarget{comments}{%
\subsection{Comments}\label{comments}}

\begin{itemize}
\item
  `How animals navigate over large-scale environments remains a riddle', first line of the abstract starts with a rather bold statement. Is this statement really true for all animals, am under the impression that there is a large body of work for at least some animals.
\item
  `\emph{We documented how young pupsdeveloped their visual-based map}' - interesting, does this mean, echolocation develops later, or that the bats are known to use primarily vision for their navigation?
\item
  Remember listening to a talk by Lee Harten in Konstanz ASAB, where she also presented work on the flight behaviour of the mother, who carried her pups around, and how the mother used to leave the pup in one tree, and so on. Do the authors discuss the implications of this type of memory on the shortcut taking ability of the young bats? \emph{Yes, the authors have taken care of this, in the SI, they clearly state the mother and the pups were brought into an indoor facility, and the pups were kept indoors until they could fly}
\item
  \emph{What about bats flying together?}, or encountering each other? Their inhourse colony data kind of excludes this idea because the individual bats arrive alone, and are spaced b a few minutes. This is not the most rigorous evidence, but is still pretty indicatibe, eg. even in Orlova Chuka (and other caves), you can see the bats arrive alone in the morning. Howeer, this still doesn't really exclude the fact that bats may be encountering each other at some point over the course of the night.
\item
\end{itemize}

\hypertarget{wikelskiearthquake}{%
\chapter{Wikelski et al.~2020}\label{wikelskiearthquake}}

\emph{Potential short-term eathquake forecasting by farm animal monitoring} \citep{wikelskiearthquake2020}

\begin{itemize}
\item
  weird animal behaviour just before earthquakes habe been reported, including dramatic cases where snakes and rats came out of their winter burrows during the winter in the 1975 Haicheng earthquake
\item
  finding reliable changes in animal behaviour is tough because animal the animal behaviour needs to be monitored before and during the earthquake.
\item
  Authors were able to overcome some of the limitations in the data this time by tagging multiple farm animals with high-resolution GPS tags that were equipped with many types of sensors
\item
  Authors measured the behaviour of animals at the M6.6 Norcia earthquake that happened on 2016
\end{itemize}

\hypertarget{methods}{%
\section{Methods}\label{methods}}

\begin{itemize}
\item
  Animals chosen from a farm based on which ones the animals thought were most sensitive to the earthquakes.
\item
  Two tagging periods, once before + during the earthquake, once after
\end{itemize}

\hypertarget{data-description}{%
\section{Data description}\label{data-description}}

\begin{itemize}
\item
  \emph{Between \ldots, the animals experienced a total of 5,304 earthquakes with M \textgreater{} 0.4 \ldots and from \ldots{} a total of 12,948} \ldots{} didn't realise that earthquakes were so frequent in some areas.
\item
  The `hypocenters' of the earthquakes were anywhere between 5-28 km from the farm --\emph{all relatively close by!}
\end{itemize}

\hypertarget{results}{%
\subsection{Results}\label{results}}

\begin{itemize}
\item
  Find a negative correlation between time of increased animal activity and earthquake intensity. For earthquakes \(\geq\) 4 M, the animals responded earlier to quakes that were closer to the farm, and later to those that were further away from the farm.
\item
  ``Warning times'' ranged from 1-15 hours
\item
  Animals seemed to be more sensitive to earthquakes in closed buildings - but there may be a seasonal factor in the observations too
\item
  What are the possible cues the animals are using to detect/respond to these earthquakes

  \begin{itemize}
  \tightlist
  \item
    The inverse relation hints at a diffusive type process. ``air ionization at pressurised rock surfaces'' -- diffusing into the air, to which the animals may be responding to.
  \end{itemize}
\end{itemize}

\hypertarget{overall-thoughts}{%
\subsection{Overall thoughts}\label{overall-thoughts}}

\begin{itemize}
\item
  very intersting paper, which quantifies something which has been known but has now been studied in greater detail through this new technology.
\item
  authors also suggest a future experimental setup where a series of animal moinorting stations could be used to predict the position and time of arrival of an earthquake.
\end{itemize}

\hypertarget{ratcliffe-et-al.-2004-can.-j.-zool.}{%
\chapter{Ratcliffe et al.~2004, Can. J. Zool.}\label{ratcliffe-et-al.-2004-can.-j.-zool.}}

\chaptermark{Conspecifics influence call design}

\emph{Conspecifics influence call design in the Brazilian free-tailed bat, Tadarida brasiliensis} \citep{ratcliffe2004conspecifics}

\begin{itemize}
\tightlist
\item
  \emph{notes taken on 2020-07-21}
\end{itemize}

\hypertarget{introduction}{%
\section{Introduction}\label{introduction}}

\begin{itemize}
\tightlist
\item
  echolocation is pretty flexible and the emitted calls vary a lot based on the type of prey being caught, and the presence of conspecifics
\item
  This paper is a kind of offspring of another (Avila-Flores 2003), where the authors saw that there was more call variation when bats flew together than when alone.
\item
  Authors try to estimate this variation by comparing observed pairs of bat calls and `virtual' pairs of bat calls.
\end{itemize}

\hypertarget{methods}{%
\section{Methods}\label{methods}}

\begin{itemize}
\tightlist
\item
  Free flying \emph{Tadarida brasiliensis} recorded in three locations in Mexico City, of bats flying over a lake, and two others in open areas (park and city square)
\item
  30 sequences from each location obtained (15 single bat passes + 15 paired passes).
\item
  Authors specifically chose call sequences with no overlap of bat calls.
\end{itemize}

\hypertarget{analysis}{%
\section{Analysis}\label{analysis}}

\begin{itemize}
\tightlist
\item
  To compare the single vs paired call behaviour, the difference in the mean values was used
\item
  The difference in the mean, \(\Delta_{mean}\) was used to compare if a bat showed alterations to its echolocation when alone vs.~when with another bats.
\item
  To understand if bats actually altered their call parameters when flying in groups the authors calculated the pair difference \(bat1_{measurement}-bat2_{measurement}\) for actual observed pairs of bat call sequences, and those of virtual pairs.
\item
  The pair difference was calculated by subtracting the means of the \(bat1_{measurement}\) and \(bat2_{measurement}\)
\end{itemize}

\hypertarget{results}{%
\section{Results}\label{results}}

\begin{itemize}
\tightlist
\item
  Authors didn't find any statistical difference in call parameters across locations and so decided to pool them all together (and thus used single bat call sequences from multiple locations while making virtual pairs). (\emph{See comments})
\item
  Authors found no inter-individual call differences across observed and virtual pairs for the parameters 1) call duration 2) bandwidth , and found differences in the peak frequency. (\emph{See comments})
\item
  The authors also saw social calls in paired audio files.
\item
  The authors are indeed careful while ending `\emph{To be even-handed\ldots.changes in peak frequency which we found\ldots are neither jamming avoidance nor air traffic control, but serve another and as yet undetermined communicative function}'
\end{itemize}

\hypertarget{comments}{%
\subsection{Comments}\label{comments}}

\begin{itemize}
\item
  authors state \emph{`Although referred to as jamming avoidance.., support for this interpretation is not as strong as that for the jamming avoidance response in electric fish'}. They also go on to state that labortatory studies seem to have `met with some success'. In general, does this support the fact that bats in the field may actually not be showing dramatic changes in calls at all - it's only when they're put into a quiet unusual lab context that they begin to show changes. Perhaps, this strong response is seen because the animals have gotten used to flying under quiest conditions, and are now suddenly challenged?
\item
  how do the authors actually know that the two bats in a paired call sequence were flying together/in close proximity? `\emph{We assinged each recorded seequence to one of two situations\ldots..and two bats flying close proximity}' - this is a rather vague definition. With a single bat detector, it is not possible to track the bats in 3d, at most one can check if the waveforms of the two bat calls are similar, but here too it only means that the two bats were flying in the same radius from the bat detector. It is somewhat crude, though admittedly, the best possible criteria given the current instrument.
\item
  Authors actually pooled call sequences from multiple locations and made virtual call pairs from this pool. This will actually have the effect of \emph{increasing} the diversity of the observed data? Also, it is known from tracking studies that individual bats tend of have favourit foraging sites - and so in some sense, it is expected that there will be a `local' flavour to the data. It would have been nice to see the authors perform the same analysis without pooling across locations.
\item
  The authors find differences in peak frequency between real and virtual pairs. This could really be an effect of where the bats were flying while in pairs vs when they were alone. The simplest explanation is that the bats may be flying further away from the microphone, thus leading to different SNRs at recording - which then leads to different peak frequencies. The difference in the peak frequency could really be an artifact and not a real effect. Moreover, given the authors state that they do see social calls, the line between social call and echolocation call is a thin one, which means, perhaps the authors are seeing this effect? The authors state this themselves \emph{`Our recordings suggest a continuum in call features between echolocation and social calls\ldots{}'}
\item
\end{itemize}

\hypertarget{benediktovadogfield}{%
\chapter{Benediktova et al.~2020}\label{benediktovadogfield}}

\emph{Magnetic alignment enhances homing efficiency of hunting dogs} \citep{benediktova2020magnetic}

\begin{itemize}
\item
  \emph{Notes taken on 2020-07-22}
\item
  `homing' behaviour has been shown in many animals, and could be driven by multiple cues. Homing behaviour in non-migratory species still needs to be excplored more.
\item
  Reports of dogs that found their way back even when displaced beyond their usual home range, and without access to familiar visual cues.
\item
  Hunting dogs have been bred by humans to chase game and then return to the start point if not followed by the hunter. These `hunts' can ve a few hundreds to thousands of meter long.
\item
  These dogs may find their way back either using their own scent trail - which the authors called `tracking' or, by actually navigating using the landmarks or information gained during the onward travel, which the authors call `scouting'
\end{itemize}

\hypertarget{methods}{%
\section{Methods}\label{methods}}

\begin{itemize}
\tightlist
\item
  The hunting dogs were left to innately follow the olfactory tracks of game animals in the wildlife reserve
\item
  Small dogs were used, and so there was no `physical threat' to wild animals
\item
  All dogs were tagged with GPS collars, and some of the trials with dogs also had a camera involved.
\item
  Trials were performed with single dogs, and in areas free of high voltage power lines, roads or buildings and across the day and season (across the year)
\item
  There needs to be some more clarity in terms of when a trial started (see \emph{Comments})
\item
  Owners hid behind a tree to reduce visual beaconing
\item
  Only excursions \textgreater200 metres were considered for this study
\item
  Authors split all excursions into 10 equal parts, and assigned phases to them based on the average speed of the part.
\end{itemize}

\hypertarget{return-strategies}{%
\subsection{Return strategies}\label{return-strategies}}

\begin{itemize}
\tightlist
\item
  Dog return strategies could be categorised into two classes

  \begin{itemize}
  \tightlist
  \item
    tracking: the inbound and outbound trajectory are the same , and the inbound-outbound trajectories were \textless30 m apart
  \item
    scouting: a new woute was taken, and the inbound-outbound trajectories were \textgreater30 m apart
  \end{itemize}
\end{itemize}

\hypertarget{results}{%
\section{Results}\label{results}}

\begin{itemize}
\tightlist
\item
  in \textasciitilde59\% of all trials, dogs ended up tracking their way back, while in 33\% of the trials, dogs ended up scouting their way back, and in 8\% of the trials, dogs used a mic of scouting and tracking
\item
  Scouting dogs showed a higher average speed while returning (because they tended to take shorter routes)
\item
  Trials bbegan and ended with no particular bias in the compass directions, though trials which were in the north-south axis, showed more efficient homing runs
\item
  wind direction and sun location didn't seem to make a difference
\item
  dogs returning with a scouting strategy seemed to begin by first travelling along the north-south axis, irrespective of the actual location of the destination
\end{itemize}

\hypertarget{discussion}{%
\section{Discussion}\label{discussion}}

\begin{itemize}
\tightlist
\item
  Authors convincingly rule out the possibility of multiple cues begin used(vision, olfaction, celestial cues). Thedogs are too short to see far, and the forest is pretty opaque. Olfactory cues are likely to change a lot, though one can't rule it out completely
\item
  The sun's location/polarised light -- this they haven't been fully able to elimination - but they indicate that there were trials where the sun wa blocked by clouds - which would reduce the strength of polarised light `map'
\item
  Authors do highlight the fact that path integration is definitely one possible mechanism by which the dogs in this tudy may be finding their way back, but the question of the north-south run still remains - and it may in fact serve as a kind of recalibration run, over which the errors accumulated over path integration can somehow be `corrected'?
\end{itemize}

\hypertarget{overall-thoughts-comments}{%
\section{Overall thoughts / Comments}\label{overall-thoughts-comments}}

\begin{itemize}
\tightlist
\item
  Really smoth introduction - easy to follow
\item
  It's not entirely clear how the trial exactly proceeded. The dogs were brought to a site, and then left to roam free.

  \begin{itemize}
  \tightlist
  \item
    \emph{The handheld GPS device was programmed to indicate when the dog had travelled ≥100 m from the position of the owner. At this moment (designated as `excursion start') the owners stopped walking\ldots{}}. This is the confusing part, because, what was the owner doing before that? Were the dog and the owner walking together (unleashed dog) on a trail, and then when the dog roamed far away enough, the trial was considered started. But this still elaves some room for error in terms, of wehere the dog left the `main trail' and where the owner was when the trial started\ldots..confusing
  \end{itemize}
\item
  In figure 4, bottom plot - authors plot the \emph{log} of the inbound track length against the raw beeline difference. This is a bit misleading perhaps? In general, yes, the log-transform will tend to show lesser variation - but in general, there will always be a linear correlation between te distance the animal travelled and the beeline distance \ldots especially if there is some kind of navigation in place\ldots{} - am I being over-cautious here or not - can't make out.
\item
  Pretty cool study, this study ads to the general observation of so many magnetically related behaviours (dogs peeing/pooping facing one direction (where did i read this), arctic foxes always facing one stereotyped direction before ponching onto their prey underneath the snow, and cows facing in one direction while grazing)
\end{itemize}

\hypertarget{habersetzermadurai}{%
\chapter{Habersetzer 1981}\label{habersetzermadurai}}

\emph{Adaptive echolocation sounds in the bat Rhinopoma hardwickei} \citep{habersetzer1981adaptive}

\begin{itemize}
\item
  \emph{Notes taken on 2020-07-22, 23}
\item
  \emph{Rhinopoma hardwickei} behaviour studied as they emerge from their caves.
\end{itemize}

\hypertarget{methods}{%
\section{Methods}\label{methods}}

\begin{itemize}
\tightlist
\item
  Speed of the bat flight measured with a stroboscipic camera
\item
  Echolocation calls recorded with a B \& K microphone placed at the cave entrance.
\item
  \emph{R. hardwickei} lives relatively close to the cave entrance, and are found with other batr species too. The bats flew at an altidtude of around 10-15metres above the ground
\item
  Bats began to return to their roosts around 3:00-3:30 in the morning, but most of them returned later between 4:30-5:45 am.
\end{itemize}

\hypertarget{results}{%
\section{Results}\label{results}}

\begin{itemize}
\item
  The author showed (I guess for the first time then) that \emph{R. hardwickei} can actually modulate its echolocation calls, ie. by showing that the calls change over the course of a landing from classical quasi-CF to very FM
\item
  During emergence, bats only emitted FM calls of \textasciitilde3ms duration
\item
  Emerging clusters seemed to have a preferred direction, while single bats that flew out didn't seem to have any preferred flight direction
\item
  Single bats also seemed to emit CF type calls, in contrast to bats that were flying in groups
\item
  Groups of bats sometimes formed over foraging grounds and in the early morning at the cave entrance itself. The bats in these groups emitted CF type calls.
\item
  Individual bats in these groups seemed to emit calls with unique CF frequencies (Fig. 4 bottom), in contrast to single bats which seem to emit at a common frequency band while flying alone (Fig. 5)
\item
  The bats seem to split thei CF frequencies into 3 bands, 30, 32.5 and 35 kHz.
\item
  \textbf{R. hardwickeii} Cf calls don't seem to have an iFM or tFM unlike rhinolophids. The author seuggests this might mean that these bats are getting time delay purely through the envelpe of the call itself. I'm also fascinated by how the laryngeal mechanism must differ between the two species. Here, the absence of a FM to start/end means the bat must release the pressure uniformly over a very short period of time.
\item
\end{itemize}

\hypertarget{overall-thoughts-comments}{%
\section{Overall thoughts / Comments}\label{overall-thoughts-comments}}

\begin{itemize}
\tightlist
\item
  Pye (1972) also found something similar to this paper apparently - need to check it out.
\end{itemize}

\hypertarget{pye-1972-j.-zool.-lond}{%
\chapter{Pye 1972, J. Zool. Lond}\label{pye-1972-j.-zool.-lond}}

\chaptermark{Bimodal CF calls in some hipposiderids}

\emph{Bimodal distribution of constant frequencies in some hipposiderid bats (Mammalia: Hipposideridae)} \citep{pye1972bimodal}

\begin{itemize}
\tightlist
\item
  \emph{notes taken on 2020-07-23} (incomplete notes, left the paper halfway through)
\end{itemize}

\hypertarget{introduction}{%
\section{Introduction}\label{introduction}}

\begin{itemize}
\item
  Bats of one species seem to show echolocation in a `common frequency band', ie. all measured bats calling within a 3kHz bandwidth
\item
  Schnitzler 1968 found a max of 2.5kHz deviation across \emph{R. ferrumequinum} and \emph{R. euryale}
\item
  Three species recorded in a cave in Kenya were however found to show `sub-bands' with no individuals calling in between two main call frequencies.
\item
\end{itemize}

\hypertarget{methods}{%
\section{Methods}\label{methods}}

\begin{itemize}
\tightlist
\item
  CF peak frequency determined through Lissajous figures (see Comments)
\end{itemize}

\hypertarget{results}{%
\section{Results}\label{results}}

\begin{itemize}
\tightlist
\item
  \emph{Hipposideros commersoni}:

  \begin{itemize}
  \tightlist
  \item
    Bat seemed to emit somewhat higher frequency than expected from its size
  \item
    Individuals either emitted at 56 or 66 kHz
  \item
    Also measuremed the inter-nostril difference, as this is known to be \textasciitilde0.5X the emitted wavelength - the nostril lengths were indeed different
  \end{itemize}
\item
  \emph{Triaenops afer}

  \begin{itemize}
  \item
    This species also showed a bimodal distribution of CF at either 79 or 88 kHz
  \item
    \emph{Here it gets interesting!} When individual animal call traces were placed over each other, then the calls seemed to be somewhat continuously spaced. But when a group of bats were flown together, bats seems to emit calls clearly in one of two sub-bands
  \item
    Fig. 2 -- could the difference in the `combined frequency profile' vs individual frequencies just be a function of flight speed. Maybe when the bats are palced together in a cage, they tend to overall fly a bit faster/slower -- leading to this difference in the emitted frequencies by individuals?
  \item
  \end{itemize}
\end{itemize}

\hypertarget{comments}{%
\subsection{Comments}\label{comments}}

\begin{itemize}
\item
  interesting to see how tricky it was back then to determine the peak CF frequency, something we do so routinely with a quick FFT nowadays. Pye had to visualise Lissajous figures, (comparing the CF call recordings with signals of known frequency )
\item
\end{itemize}

\hypertarget{jones-ransome-1993-proc.-r.-soc.-lond.-b}{%
\chapter{Jones \& Ransome 1993, Proc. R. Soc. Lond. B}\label{jones-ransome-1993-proc.-r.-soc.-lond.-b}}

\chaptermark{R. ferumequinum CF stability}

\emph{Echolocation calls of bats are influenced by maternal effects and change over a lifetime} \citep{jonesransome1993}

\begin{itemize}
\tightlist
\item
  \emph{notes taken on 2020-07-23}
\end{itemize}

\hypertarget{introduction}{%
\section{Introduction}\label{introduction}}

\begin{itemize}
\tightlist
\item
  \emph{R. ferrumequinum} emits CF calls of \textasciitilde50ms long at 83kHz.
\end{itemize}

\hypertarget{methods}{%
\section{Methods}\label{methods}}

\begin{itemize}
\tightlist
\item
  Ringed wild bats were captured and recorded as they echolocated. Overall, 386 bats were recorded in the hand.
\item
  Bats ranged from 1-28 years old
\end{itemize}

\hypertarget{results}{%
\section{Results}\label{results}}

\begin{itemize}
\tightlist
\item
  Age seemed increase the resting frequency (RF), with a high increase from yaer 1 to 2, and then a gradual increase.
\item
  There was also a seasonal effect - where the lowest RFs were recorded during the winter, and highest values in summer.
\item
  The RF of infant bats were correlated with the RF of the mother (See comments)
\item
  The authors also find a correlation with the age of the mother and the sex of the infant on its RF. Pups born to younger mothers had higher RFs, than those born to older mothers
\end{itemize}

\hypertarget{discussion}{%
\section{Discussion}\label{discussion}}

\begin{itemize}
\tightlist
\item
  authors point out that changes in RF with season may have to do with body temperature - as even active bats may not be as warm as they are in the summer.
\item
  The correlation between mother and pup RFs may be explained by some form of learning perhaps
\item
  changes in RF over age could be the result of hearing loss (neural/cochlear)
\end{itemize}

\hypertarget{comments}{%
\subsection{Comments}\label{comments}}

\begin{itemize}
\tightlist
\item
  the x and y axed of Fig. 3 aren't equal make it hard to judge visually how strong the correlation is. Numbers-wise at least, the \emph{r} value isn't particularly high.
\item
  back in 1993, the processing times were quick! The paper was received 1 Feb.~1993, and accepted on 16 Feb.~1993!!
\end{itemize}

\hypertarget{schuchmann-siemers-2010-plos-one}{%
\chapter{\texorpdfstring{Schuchmann \& Siemers 2010, \emph{PLoS One}}{Schuchmann \& Siemers 2010, PLoS One}}\label{schuchmann-siemers-2010-plos-one}}

\chaptermark{Horseshoe bat source levels in Bulgaria}

\emph{Variability in Echolocation Call Intensity in a Communityof Horseshoe Bats: A Role for Resource Partitioning orCommunication?} \citep{schuchmannsiemers2010a}

\begin{itemize}
\tightlist
\item
  \emph{notes taken on 2020-07-23,24}
\end{itemize}

\hypertarget{introduction}{%
\section{Introduction}\label{introduction}}

\begin{itemize}
\item
  Authors measure the source levels of a whole community of horseshoe bats near and around the Tabachka field station. They record call levels of 5 species.
\item
  The CF distributions of the four-five species that geographically overlap in Europe, also overlap spectrally
\item
  \emph{Rhinolophus mehelyi} can tell apart conspecific calls in the midst of other overlapping species (See comments)
\item
  Do some bats (that are calling at higher than allometrically predicted frequencies) evolve towards a `private frequency' channel?
\item
  Authors look at whether \emph{R. mehelyi} produce louder calls to compensate for decrease in detection distance because they call at a higher call frequency for their given body size (\protect\hyperlink{com_shuchsiem}{Comments 2})
\item
  The questions the authors ask are:

  \begin{itemize}
  \tightlist
  \item
    Are the call intensitites different across species, and could call intensities play a role in nice differentiation (\protect\hyperlink{com_shuchsiem}{Comments 3})
  \item
    Does \emph{R. mehelyi}, the higher-than-allometrically predicted species, call at `especially high' call intensities? (\protect\hyperlink{com_shuchsiem}{Comments 4})
  \item
    How does max. call intensity vary with body size, sex -- and could it be used as an honest signal bats can use to identify conspecifics (\protect\hyperlink{com_shuchsiem}{Comments 5})
  \end{itemize}
\end{itemize}

\hypertarget{methods}{%
\section{Methods}\label{methods}}

\hypertarget{study-animals}{%
\subsection{Study animals}\label{study-animals}}

\begin{itemize}
\tightlist
\item
  \emph{R. mehelyi, euryale, ferrumequinum, hipposideros} caught from the area around the field station, \emph{R. blasii} caught from the Eastern Rhodopes.
\item
  Bats recorded in a large-ish room (8x4x2.5m) (\protect\hyperlink{com_shuchsiem}{Comments 4}), and can thus be expected to emit louder calls than when in a cluttered environment
\end{itemize}

\hypertarget{call-intensitites-and-acoustic-analysis}{%
\subsection{Call intensitites and acoustic analysis}\label{call-intensitites-and-acoustic-analysis}}

\begin{itemize}
\item
  Done with a B \& K mic placed 1m away fromt he bat's head - \emph{this is a pretty neat and direct way to do it because this is a perching bat anyways!!} \ldots{} though, the clutter caused by the mic being right in front of the bat \ldots hmmm\ldots{}
\item
  There;s a large variation in the frequency resolution across the years. It also is a bit unclear whether the authors used an automatic or manual method -- though my current reading seems to suggest a semi-manual method to estimate the peak frequency of a call
\item
  To get call intensity, the authors seem to do a lot of call processing.

  \begin{itemize}
  \tightlist
  \item
    FIR bandpass filter , with a 128 order bandpass filter(\protect\hyperlink{com_shuchsiem}{Comments 6}) and then proceeding to take the FFTs within the call\ldots.why do this when you can take the waveform value directly??)
  \end{itemize}
\item
\end{itemize}

\hypertarget{results}{%
\section{Results}\label{results}}

\hypertarget{overlap-of-frequency-bands}{%
\subsection{Overlap of frequency bands}\label{overlap-of-frequency-bands}}

\begin{itemize}
\tightlist
\item
  species showed overlap in their CF frequencies

  \begin{itemize}
  \tightlist
  \item
    \emph{R. mehelyi, euryale, hipposideros} overlapped, while those of \emph{R. ferrumequinum} and \emph{R. hipposideros} were different
  \end{itemize}
\item
  Statistically, the three overlapping species differed (but the sample sizes weren't too big either..)
\end{itemize}

\hypertarget{call-parameters}{%
\subsection{Call parameters}\label{call-parameters}}

\begin{itemize}
\tightlist
\item
  No observed correlation between call intensity and call frequency
\item
  \emph{R. euryale} seemed to call 10-17 dB lower than the other species.
\item
  Intra-individual variation of 0-334 Hz across species
\item
  Intra-individual call intensity variation also of between 2.3 to 5.5 dB
\end{itemize}

\hypertarget{discussion}{%
\section{Discussion}\label{discussion}}

\begin{itemize}
\tightlist
\item
  authors suggest call intensity can be an informative cue that can be used to assess species identity. They propose that perching bats may be able to reconstruct the position and `head-aim' of the other bat, and thus assess its source-level. \emph{While it may be informative to tell apart }R. mehelyi*, which calls fainter than others - there may actually be other cues eg. FM sweeps, or even more `voice' type characteristics embedded in the whole call that could be picked up by bats themselves?)
\end{itemize}

\hypertarget{com_shuchsiem}{%
\subsection{Comments}\label{com_shuchsiem}}

\begin{itemize}
\item
  \begin{enumerate}
  \def\labelenumi{\arabic{enumi}.}
  \tightlist
  \item
    Need to read ref. 24, Schuchmann \& Siemers 2010 (Am. Nat.) - a habituation-dishabituation study
  \end{enumerate}
\item
  \begin{enumerate}
  \def\labelenumi{\arabic{enumi}.}
  \setcounter{enumi}{1}
  \tightlist
  \item
    How does the \emph{allometry} matter per se. The only thing that will matter for the detection distance is the source level and the emitted frequency (unclear)
  \end{enumerate}
\item
  \begin{enumerate}
  \def\labelenumi{\arabic{enumi}.}
  \setcounter{enumi}{2}
  \tightlist
  \item
    It seems a bit tenuous, and obvious- but it's also hard to show a connection well?. Yes, low source level bats may only be able to detect insects from nearby, and high SL bats can detect insects from far away -- not sure, but it sounds vague
  \end{enumerate}
\item
  \begin{enumerate}
  \def\labelenumi{\arabic{enumi}.}
  \setcounter{enumi}{3}
  \tightlist
  \item
    I don't necessarily see a reason for it to call at higher source levels than what other species with similar CF do\ldots{} not sure, why it would call at `especially high' call intensities.
  \end{enumerate}
\item
  \begin{enumerate}
  \def\labelenumi{\arabic{enumi}.}
  \setcounter{enumi}{4}
  \tightlist
  \item
    Probably this is the `big flight room' at Tabachka?
  \end{enumerate}
\item
  \begin{enumerate}
  \def\labelenumi{\arabic{enumi}.}
  \setcounter{enumi}{5}
  \tightlist
  \item
    Vaguely remember how using a very high order filter can mess with the signal structure itself (altering phases etc.). If the signal structure itself is altered so much, then what is the guarantee that the peak eq. source level obtained in this study is valid? If anything at all, the call intensitites reported in this study may actually be at the lower end of what is known because of the heavy filtering that has been done?
  \end{enumerate}
\item
  \begin{enumerate}
  \def\labelenumi{\arabic{enumi}.}
  \setcounter{enumi}{6}
  \tightlist
  \item
    Odd grammar and typos in a bunch of places.
  \end{enumerate}
\end{itemize}

\hypertarget{halsey-2019-biology-letters}{%
\chapter{\texorpdfstring{Halsey 2019, \emph{Biology Letters}}{Halsey 2019, Biology Letters}}\label{halsey-2019-biology-letters}}

\chaptermark{Alternatives to *p*-value testing}

\emph{The reign of the p-value is over: what alternative analyses could we employ to fill the power vacuum?} \citep{halsey2019a}

\begin{itemize}
\tightlist
\item
  \emph{notes taken on 2020-07-24}
\end{itemize}

\hypertarget{introduction}{%
\section{Introduction}\label{introduction}}

\begin{itemize}
\item
  The p value has been used and abused, and in some sense has contributed to the reproducibility crisis in science
\item
  The p value is based on the null hypothesis being true, and therefore doesn;t actually provide any information on how true the alternate hypothesis is -- which is actually our question.
\item
  If the p-value is `high'/non-significant -- *it doesn't tell us anything because there is an open question if our method could pick up a difference given the current sample size?
\item
  \textbf{`\emph{Moreover, with a big enough sample size, inevitably the null hypothesis will be rejected; perversely, a p-value based statistical result is as informative about our sampole as it is about our hypothesis}'} (See comments)
\item
  p-values can vary between replicates of the same study even when statistical power is high! (See comments)
\item
  interpreting p-values as signficant or not (dichotomously) encourages failed experiment replication --- hmmm
\item
  Even with 80\% power ---\textgreater{} prob. of getting two studies/replicates significant is 0.64 (0.8\(^{2}\)), while getting even one of these two studies/replicates significant is 0.32(\(2\times0.8\times0.2\))
\item
  \textbf{`\emph{th ep-value is typically highly imprecise about theamount of evidence against the null hypothesis, and thuspshould be considered as providing only loose, first passevidence about the phenomenon being studied}'}
\end{itemize}

\hypertarget{alternates-to-p-value-testing}{%
\section{Alternates to p-value testing:}\label{alternates-to-p-value-testing}}

\hypertarget{p-value-prediction-interval}{%
\subsection{P-value prediction interval}\label{p-value-prediction-interval}}

\begin{itemize}
\tightlist
\item
  the p-value is attractive, and a nice objective indicator of evidence against the null hypothesis
\item
  author suggests presenting \emph{variability} of p-values across samples, because p-values can be so variable : the `p-value prediction interval' \href{http://methods.sagepub.com/journal-article/sage-quantitative-research-methods/replication_and_p_intervals_p_values_predict_the_future_only_vaguely_but_confidence_intervals_do_much_better/d36.xml}{Cumming 2008}
\end{itemize}

\hypertarget{estimate-likelihood-of-false-positive}{%
\subsection{Estimate likelihood of false positive}\label{estimate-likelihood-of-false-positive}}

\begin{itemize}
\tightlist
\item
  This method requires a prior `feeling'/idea of how likely it is that there will be an effect , this is kind of hard to come by for observational studies, or those where there's no literature to say anything ??
\item
  The author also provides an `inverse' option, where the inputs are the obtained p-value + the power of the test. If the false positive risk is set at 5\% (\(\alpha=0.05\)), then the tool will provide an estimate of your prior expectation that the treatment will have an effect
\end{itemize}

\hypertarget{effect-sizes-and-confidence-intervals}{%
\section{Effect sizes and confidence intervals}\label{effect-sizes-and-confidence-intervals}}

\begin{itemize}
\tightlist
\item
  The effect size (differnce in mean between the groups) is easy to estimate - and a 95\% confidence interval can be reported along with it.
\item
  The confidence interval based approach with full reporting of the info prevents false claims from being made (yes/no effect)
\end{itemize}

\hypertarget{bayes-factor-comparitive-evidence-of-the-null-and-alternate-hypotheses}{%
\section{Bayes factor: comparitive evidence of the null and alternate hypotheses}\label{bayes-factor-comparitive-evidence-of-the-null-and-alternate-hypotheses}}

\begin{itemize}
\tightlist
\item
  the controversy about the `subjectivity' in defining the prior can be solved by running a sensitivity analysis that explores a range of prior definitions
\item
  the `simplified Bayes factor' (SIB) is an alternate index which can be used to assess (Seems to be more relevant for planned studies and their stopping rules, also see Comments)
\end{itemize}

\hypertarget{akaike-information-criterion}{%
\section{Akaike information criterion}\label{akaike-information-criterion}}

\begin{itemize}
\tightlist
\item
  The AIC provides an idea of how well your model represents reality - and handles the tradeoff berween model complexity and over-fitting
\item
  AIC is nice to compare between models, but it doesn't provide an absolute estimate of which model is the best. Among the models that are under consideration, it may very well be the case that the one with the lowest AIC doesn't capture the variation in the data as well as the other models - a check of the model's actual fit still needs to be done.
\end{itemize}

\hypertarget{conclusion}{%
\section{Conclusion}\label{conclusion}}

\begin{itemize}
\tightlist
\item
  \textbf{`\emph{Primarily, the argument goes, you should prioritize interpretation of graphical plots of your data, where possible, and treat statistical analyses as supporting or confirmatory information}'}
\end{itemize}

\hypertarget{comments}{%
\subsection{Comments}\label{comments}}

\begin{itemize}
\tightlist
\item
  A low p-value can thus be a result of a high effect size for a given sample size, or a very large sample size for a given small effect size. So, how does one go about disentangling this issue then?? Perhaps with power analyses -- which then adds to the amount of work that needs to be done anyway..?
\item
  the p-value is not stable\ldots hmm, even across replicates - this is kind of scary to know. ie. in \citep{halsey2015fickle}, they state \emph{`The reason for this is that unless statistical power is very high, the P value exhibits wide sample-to-sample variability and thus does not reliably indicate the strength of evidence against the null hypothesis'}, and \emph{`If statistical power is limited, regardless of whether the P value returned from a statistical test is low or high, a repeat of the same experiment will likely result in a substantially different P value'}
\item
  Not sure how different this simplified Bayes factor is from a p-value - I also don't understand how it overcomes the problems associated with dichotomous type p-value testing. Moreover, if the p-value itself is `fickle', then it will also result in unstable SIB estimates too -- how to overcome this problem\ldots{}
\item
  haha, nice - the author also provides a link to another paper \href{https://jeb.biologists.org/content/220/17/3007}{Sneddon, Lewis, Bury 2017, J. Exp. Biol.}, where Box 2 has a template passage highlighting the problems of p-values!
\end{itemize}

\hypertarget{halsey-curran-everett-vowler-drummond-2015-nature-methods}{%
\chapter{\texorpdfstring{Halsey, Curran-Everett, Vowler, Drummond, 2015 \emph{Nature Methods}}{Halsey, Curran-Everett, Vowler, Drummond, 2015 Nature Methods}}\label{halsey-curran-everett-vowler-drummond-2015-nature-methods}}

\chaptermark{Fickle *p* values}

\emph{The fickle P value generates irreproducible results} \citep{halsey2015fickle}

\begin{itemize}
\tightlist
\item
  \emph{notes taken on 2020-07-24}
\end{itemize}

\hypertarget{introduction}{%
\section{Introduction}\label{introduction}}

\begin{itemize}
\tightlist
\item
  p-values often used without understanding the statistical power in play
\item
  statistical power is often only considered after an `effect' could not be detected
\item
  unless statistical power is very high, p-values vary a lot from sample to sample
\item
  Ronald Fisher actually intended p-values to be used as a continuous indication of support for the null hypothesis

  \begin{itemize}
  \tightlist
  \item
    but, here too - if the statistical power is already low, no matter what - the p-value will not be reliable!!
  \end{itemize}
\end{itemize}

\hypertarget{statistical-power-is-generally-low-p-values-cant-be-trusted}{%
\section{statistical power is generally low --\textgreater{} p values can't be trusted}\label{statistical-power-is-generally-low-p-values-cant-be-trusted}}

\begin{itemize}
\tightlist
\item
  Even when statistical power is close to 90\%, p-values are not stable
\item
  Considering that many studies are designed with 80\% power, this means there is probably even more variation in p-values to be expected
\item
  low statistical power -----\textgreater{} low sample sizes --\textgreater{} higher data variability
\end{itemize}

\hypertarget{exaggerated-effect-sizes}{%
\section{Exaggerated effect sizes}\label{exaggerated-effect-sizes}}

\begin{itemize}
\tightlist
\item
  A low-power test will give a low p-value only when the samples are very unrepresentative (ie. come from the edges of the distribution) -- but it can definitely happen (Comments)
\end{itemize}

\hypertarget{comments}{%
\subsection{Comments}\label{comments}}

\begin{itemize}
\item
  the possibility of extreme samples contributing to low p-values, may also lead to a dramatic effect size. In some sense, there may be no \emph{statistical} way to overcome this problem - the only way to resolve this situation may be to increase sample size?
\item
\end{itemize}

\hypertarget{li-et-al.-2014-behav-ecol-sociobiol-incomplete}{%
\chapter{\texorpdfstring{Li et al.~2014, \emph{Behav Ecol Sociobiol} (Incomplete)}{Li et al.~2014, Behav Ecol Sociobiol (Incomplete)}}\label{li-et-al.-2014-behav-ecol-sociobiol-incomplete}}

\chaptermark{Behavioural responses to sympatric echolocation calls}

\emph{Behavioral responses to echolocation calls from sympatric heterospecific bats: implication for interspecific competition} \citep{li2014behavioral}

\begin{itemize}
\tightlist
\item
  \emph{notes taken on 2020-07-27}
\end{itemize}

\hypertarget{introduction}{%
\section{Introduction}\label{introduction}}

\begin{itemize}
\tightlist
\item
  sympatric species with simlar diets and ecologies could share information and gain mutual benefit through this process. Mutual recognition is required for mutual benefit to occur.
\item
  agression between species at the same trWophic level should be selected against by natural selection as it increases the risk of predation
\item
  In free-flying bats, species recognition has been shown to be related to simlarity in call structure and diet ecologies
\item
  \emph{`Interspecific recognition may thrterefore be necessary to adjust the strength of interspecific interactions'} - what is meant by this line - adjusted by whom?
\item
  the trophic niche index - how `linear'/uniform is it as an index?
\item
  Authors tested which bats responded to which bats, and checked if call similarity and trophic overlap explained the observed correlation
\item
  3 species were tested, among the 4 species that are commonly found together:

  \begin{itemize}
  \tightlist
  \item
    \emph{Rhinolophus macrotis}
  \item
    \emph{R. macrotis}
  \item
    \emph{R. lepidus}
    All the rhinolophid species were observed in the expeirments --but there were no \emph{Asellia}'s in this experiment. ( \emph{Asellia stoliczkanus})
  \end{itemize}
\end{itemize}

(\protect\hyperlink{com_libehav}{Comments 1})

\hypertarget{methods}{%
\section{Methods}\label{methods}}

\begin{itemize}
\item
\end{itemize}

\hypertarget{results}{%
\section{Results}\label{results}}

\hypertarget{discussion}{%
\section{Discussion}\label{discussion}}

\hypertarget{com_libehav}{%
\subsection{Comments}\label{com_libehav}}

\hypertarget{ming-et-al.-2020-pnas}{%
\chapter{\texorpdfstring{Ming et al.~2020, \emph{PNAS}}{Ming et al.~2020, PNAS}}\label{ming-et-al.-2020-pnas}}

\chaptermark{Modelling frequency hopping in echolocation}

\emph{How frequency hopping suppresses pulse-echo ambiguity in bat biosonar} \citep{Ming17288}

\begin{itemize}
\tightlist
\item
  \emph{notes taken on 2020-07-28}
\end{itemize}

\hypertarget{introduction}{%
\section{Introduction}\label{introduction}}

\begin{itemize}
\tightlist
\item
  In cluttered environments, echoes that arrive after the next call can cause call-echo ambiguity.
\item
  When interpulse interval (IPI) is \textgreater{} `echo-epoch' (time taken for the last echoes in the acoustic scene to arrive, \(\alpha\) `depth' of the scene) - then there is no ambiguity
\item
  Echoes from the previous call that arrive can appear to be objects that are very close by.
\item
  However, in cluttered conditions, bats reduce their IPI to keep up the update rate, which means the chance of such ambiguities is higher.
\item
  `Frequency hopping' helps to overcome this call-echo matching problem.
\item
  \emph{Eptesicus fuscus} shows shifts of upto 5-7 kHz between one call to the next while doing this kind of echo hopping. However, even a shift of 5-7 kHz means an overlap of 70-80\% from one call to the other (\protect\hyperlink{com_ming}{Comments 1})
\end{itemize}

\hypertarget{quick-primer-on-echo-delay-estimation}{%
\subsection{Quick primer on echo delay estimation}\label{quick-primer-on-echo-delay-estimation}}

\begin{itemize}
\tightlist
\item
  In manmade ranging systems (radar/sonar), all frequencies in a wideband sweep contribute equally to the matched filtering.
\item
  In \emph{E. fuscus}, the first harmonic is required for echoes need to have a first harmonic to be `registered', but the second harmonic (FM2) plays a stronger role in modulating echo delay perception.
\item
  Authors test the specific idea that even \emph{within} the first harmonic itself, \emph{E. fuscus} only needs a limited region of the FM1 to detect echoes.
\end{itemize}

\hypertarget{methods}{%
\section{Methods}\label{methods}}

\begin{itemize}
\item
  Authors test if 1) only the lowest paer of FM1 frequencies are required for delay perception 2) unclear what is being \emph{tested} here..\emph{`Second, the small, seemingly not very effective, size of the frequency hopping by just a few kilohertz \ldots.suggests that the hypothesized necessity of these lowest FM1 frequencies may be the key to why the harmonics are processed asymmetrically'}
\item
  More and more of the lower frequencies were filtered out and the bats were asked to detect the S+ (\protect\hypertarget{com_ming}{}{Comments 3})
\item
  If the authors want to understand which parts of the echo are important - why do they focus on an AFC (alternate forced choice) experiment where the task is to tell apart two echoes - with different arrival delays \emph{and} different structures (single glint vs double glint) (\protect\hypertarget{com_ming}{}{Comments 4})
\item
  It seems like \emph{both} phantom echoes were high/low passed filtered - but this isn't explicityly stated in the paper!
\item
  4 \emph{E. fuscus} were trained in an alternate force choice experiment
\item
\end{itemize}

\hypertarget{results}{%
\section{Results}\label{results}}

\begin{itemize}
\tightlist
\item
  the region between 29-30 kHz is important for the bats to tell apart S+ and S-
\item
  Confusing \emph{`Finally, when all frequencies are removed from S+ by low-pass filtering at 20 kHz, there is no two-glint S+ at all,..'} - if `all' frequencies are removed - there is no echo at all in principle, fo
\item
  `\emph{..vast proportion of frequencies do not support perception of echo delay unless the critical frequencies of 25 to 30 kHz are present.}'
\item
  authors also cite previous studies where when FM bats were faced with CF playbacks, they responded strongly to playbacks between 24-28 kHz, and didn't respond at all when other frequencies were played back.
\end{itemize}

\hypertarget{discussion}{%
\section{Discussion}\label{discussion}}

\begin{itemize}
\tightlist
\item
  The authors seem to discuss the implications of the last few kilohertz results into how frequency hopping works -- they haven't actually shown this directly through the experiments. The data only says the last few kilohertz is important for echo delay detection.
\item
  The results indicate that the last few kilohertz is important - but at the same time don't point to a mechanism which allows the bat to detect echoes which are frequency shifted - how would the bat register the fact that the `terminal frequency' has moved up after it's shifted its echolocation? This is not really discussed in the paper directly ?..
\end{itemize}

\hypertarget{the-scat-model}{%
\section{the SCAT model}\label{the-scat-model}}

\begin{itemize}
\tightlist
\item
  the SCAT model is a bio-inspired model which aims to mimic how the auditory system of bats work.
\item
  delay perception is done by parallel bandpass filters which detect the time between sounds with energy in their bandpass region. Thus, they are able to tell the delay between call and echo fairly accurately.
\item
  The idea the authors propose is essentially that the lower part of each call is the one that triggers echo detection.
\item
  Imagine one call with terminal frequency at 20 kHz, and the next call with the terminal frequency at 25 kHz. Even if an echo from the previous call were to arrive - it'd have much lower frequencies than that of the current - call and wouldn't be `sent' into the processing chain. (\protect\hypertarget{com_ming}{}{Comments 5})
\end{itemize}

\hypertarget{com_ming}{%
\subsection{Comments}\label{com_ming}}

\begin{enumerate}
\def\labelenumi{\arabic{enumi}.}
\tightlist
\item
  How is this overlap actually calculated - it can be done in multiple ways\ldots including/excluding the second harmonics etc..? As the authors themselves mention later - a shift of 5-7 kHz in the first harmonic corresponds to a shift of 10-14 kHz in the second harmonic.
\item
  Fig. 2 is the first figure that is referred in the paper (and the one that brings the phenomenon to the reader directly) - but, Fig.1 (related to a more technical point) is presented first - could have been rearranged.
\item
  S+ is introduced without any kind of explanation, and the figure is the main reference for the paper. This is kind of understandable given the word limits, but it's also kind of unfortunate because the reader's flow is interrupted each time by having to go back and forth between sections of the paper.
\item
  I'm not able to understand why the authors trained the bats to distinguish two echoes that differ in two characteristics. Am wondering how this would affect what can actualy be said from the experiments. ie. the results could also be narrowly interpreted to say, `bats require the lowest parts of their FM1 to tell apart a double glint echo from a single glint echo' \ldots{} It would have been cleaner if the authors had actually compared the bats ability to tell apart two echoes with the same structure, but arriving at different delays -- this would actually say something about what is required to detect an echo, rather than a double glint vs a single glint echo. The task is made \emph{easier} by the introduction of such different properties \emph{`..question is whether this easy task still is easy (sic) if the lowest fre quencies in FM1 are removed from the S+ echoes'}, but the point is that it may also muddle the interpretation?
\item
  There seems to be a disconnect between what was done and what is being discussed. What was shown is that the end frequencies are important for echo discrimination (the glint structure X different delays confuses things according to me). But what is being discussed is how the frequency hopping is reliant on the lower frequencies of each call - the authors haven't really `shown' this experimentally - but suggest a model which may explain it. There's something that I can't put into words better - need to give this a second try again.
\item
  The authors don't show the actual double glint echo - which is somewhat odd too.
\end{enumerate}

\hypertarget{additional-comments-from-afeg-paper-primer-meeting-on-2020-08-04-leonie-paula-holger-thejasvi}{%
\subsection{Additional comments from AFEG Paper primer meeting on 2020-08-04 (Leonie, Paula, Holger, Thejasvi)}\label{additional-comments-from-afeg-paper-primer-meeting-on-2020-08-04-leonie-paula-holger-thejasvi}}

(Leonie leading, and other group members commenting)

\begin{itemize}
\item
  As the authors state, when the low pass is at 20kHz -there should be no
  double glint echo (silence on one side), which means the bats only choose the double glint,
  instead of just reverting to the side with one echo. This seems kind of odd
\item
  The bat could actually be using frequency based cues, rather than
  the delay
\item
  Not so many experimental details - a supplementary information could have been included
\item
  The results only say that the last few kilohertz is important for telling apart S- and S+ - it doesn't necessarily
  say whether this part plays a role in glint enumeration (single vs double) or delay acuity!
\item
  Authors haven;t really shown that in general the last few kilohertz aer important between calls, but the authors here have only shown that
  25-30 kHz is important.
\item
\end{itemize}

\hypertarget{lakens-daniel-blog-post-from-the-20-statistician}{%
\chapter{\texorpdfstring{Lakens, Daniel, blog post from \emph{`The 20\% statistician'}}{Lakens, Daniel, blog post from `The 20\% statistician'}}\label{lakens-daniel-blog-post-from-the-20-statistician}}

\chaptermark{Adjusting alpha for statistical tests}

\emph{Why you don't need to adjust your alpha level for all tests you'll do in your lifetime} \citep{daniellakensa_2016}

\begin{itemize}
\tightlist
\item
  \emph{notes taken on 2020-08-04}
\end{itemize}

\hypertarget{introduction}{%
\section{Introduction}\label{introduction}}

\begin{itemize}
\item
  Take a case where people are randomly assigned into two groups, and two unrelated dependent variables are measured. The groups are compared. True positive of no effect on both the variables is: 0.95 X 0.95 = 0.9025. Probability of at least one of the variables showing up as significantly different is: 1 - 0.9025 = \textbf{0.0975} !! How to control such erroneous false positives?
\item
  There are a bunch of methods which can control the `error rates':

  \begin{enumerate}
  \def\labelenumi{\arabic{enumi}.}
  \tightlist
  \item
    Bonferroni correction
  \item
    Holm-Bonferroni sequential procedure
  \end{enumerate}
\item
  When the \# of statistical tests increase, it may be better to control false discovery rate, rather than error rates.
\end{itemize}

\hypertarget{the-bonferroni-correction}{%
\section{The Bonferroni correction}\label{the-bonferroni-correction}}

\begin{itemize}
\item
  \emph{`The Bonferroni correction controls the familywise error rate..'} \protect\hypertarget{lakens_com}{}{\emph{what is family wise error rate - See} Comments}
\item
  \emph{`..is not straight forward is that error control does not just aim to control the number of erroneous statistical inferences, but the number of erroneous theoretical inferences.'}. Is the author trying to say that corrections need to be done when a bunch of tests are trying to answer the same question, with a common dataset - don't fully understand
\item
  Experiment wise Type I error rates (\emph{I guess this is the famous p value?}) can be calculated from the setup of the stats? The example given by the author here is in a 2x2x2 ANOVA (See \protect\hypertarget{lakens_com}{}{Comments 2}) - there are seven tests that are done, and so when a 0.05 \(\alpha\) is used each time, this make the probability of getting at least one effect at least 30\%. \emph{(\protect\hypertarget{lakens_com}{}{Comments 3})}
\item
  The author also gives a more experimental example. The case of comparing predictions from two theories. One theory predicts interaction in a 2x2 ANOVA, and the other doesn't predict an interaction, but at least one main effect. (A 2x2 ANOVA tests 3 null hypotheses, two main effects and one interaction).\\
\item
  A naive way to approach the Bonferroni correction is to set the \(\alpha\) to \(\frac{\alpha}{3}\)
\item
  But here the author points out that because theoretically the two theories predict different outcomes, the \(\alpha\) also needs to corrected separately. Since Theory B predicts at least one significant main effect (out of 2 main effects) - we can set the new \(\alpha\) to \(\alpha /2\) and so if the p \textless{} \(\alpha/2\) we will accept Theory B
\item
  Since Theory A predicts only one interaction being significant, we don't really need to correct for anything. (\protect\hypertarget{lakens_com}{}{Comments 4})
\item
  Maintaining a sensible \(\alpha\) value with multiple testing corrections allows you to be wrong at most at the expected \(\alpha\) rate.
\item
  Should you correct for all the tests you will do over your lifetime? Yes, but only if all the data/work you've done is going into testing one single theory !
\item
  \emph{`..criticism on corrections for multiple comparisons is that it is strange that the conclusions a researcher draws depends on the number of additional tests a researcher performs.'} - here the authors cites the case of a 2 group comparison being significant at 0.05 (p=0.04), but when including a second dependent variable, the \(\alpha\) is now 0.025, which means the first variable data no longer rejects the null hypothesis. ..\emph{`Lowering alpha levels is a mathematical necessity when you want to control error rates, but it is not needed if ..you..quantify relative likelihoods of the data under different hypotheses.'}. Does the author mean that the comparison of the two groups for each of the dependent variables corresopnds to two independent null hypotheses? In what way is this different from the 2x2x2 ANOVA that was just discussed before?
\item
  You can also increase the alpha, and sometimes this makes sense to do so - ie. while pre-registering two studies, you can set the alpha at 0.2236, and decide that only if \emph{both} studies show \(p<0.2236\) will the null hypothesis be rejected. This is effectively the same as setting the overall \(p\) at 0.05 because 0.2236X0.2236 = 0.05 - {[}Comments 5{]}
\item
  \emph{`There is only one reason to calculate p-values, and that is to control Type 1 error rates using a Neyman-Pearson approach.'} - the focus here is on controlling false positives.
\end{itemize}

\hypertarget{lakens_com}{%
\section{Comments/Questions}\label{lakens_com}}

\begin{enumerate}
\def\labelenumi{\arabic{enumi}.}
\tightlist
\item
  what is a `familywise error rate': The probability of making at least on Type I error (False Positive): \(FWE \leq 1-(1-\alpha_{test})^{Number\ of\ comparisons}\). eg. at an \(\alpha\) of 0.05 and with 10 tests, the \(FWE\) is : \(\leq (1-0.05)^{10}=0.401\) (thanks to this \href{https://www.statisticshowto.com/familywise-error-rate/}{page})
\item
  2x2x2 ANOVA is when there are three variables and two groups in each variable. ie. imagine you're testing a drugs efficacy, and you have three variables (Thanks to this \href{https://www.graphpad.com/guides/prism/7/statistics/stat_what_is_three-way_anova_used_f.htm}{page}) : sex, dosage level and treatment/control. Within Sex there's male and female, within dosage level there's high and low, and then there's treatment/control. There are seven p-values because it performs a series of comparisons within the 2x2x2 complex:

  \begin{itemize}
  \tightlist
  \item
    3 tests comparing the groups within each of the variables (eg. male vs female, treatment vs control, high vs low)
  \item
    3 tests testing two-way interactions: pool data from Sex, effect of treatment vs control in low and high dose groups
  \item
    1 test on three way interaction: \emph{`there is no three way interaction among all three factors'}
  \end{itemize}
\item
  I get this example, within one statistical test, there are seven comparisons being made - and when each of these have an alpha value, there can be a false positive - what about when you are performing single comparisons of multiple variables from two groups?
\item
  I guess one of the main issues that the author is trying to bring out is that by naively correcting and setting \(\alpha\) values very low, we risk running into Type II (false negative) errors! By setting
\item
  What would have happened if the alpha was still set at 0.05 for both studies, then the effective false positive rate would have been 0.025 -- which is perhaps too conservative? Also, with an alpha set at 0.2236, the probability of a false negative is:
\end{enumerate}

\hypertarget{authors-et-al.-yyyy-j.-sthing.-sthing.}{%
\chapter{\texorpdfstring{AUTHORS ET AL. YYYY, \emph{J. STHING. STHING.}}{AUTHORS ET AL. YYYY, J. STHING. STHING.}}\label{authors-et-al.-yyyy-j.-sthing.-sthing.}}

\chaptermark{SHORT TITLE HERE}

\emph{BIG TITLE OUT HERE IN FULL} \citep{schuchmannsiemers2010a}

\begin{itemize}
\tightlist
\item
  \emph{notes taken on YYYY-MM-DD}
\end{itemize}

\hypertarget{introduction}{%
\section{Introduction}\label{introduction}}

\hypertarget{methods}{%
\section{Methods}\label{methods}}

\hypertarget{results}{%
\section{Results}\label{results}}

\hypertarget{discussion}{%
\section{Discussion}\label{discussion}}

\hypertarget{comments}{%
\subsection{Comments}\label{comments}}

  \bibliography{book.bib}

\end{document}
